% \documentclass[11pt]{article}
% article: for articles, presentations, short reports, documentation.
% report: for reports containg several chapters, small books, thesis
% book: for real books
% letter: for writing letters

% \usepackage{amsmath}
% amsmath, amssymb and amsthm: for math symbols
% bable: internationalization
% fontenc: font encoding
% geometry: easy management of document margins
% graphicx: manage external pictures
% inputenc: encoding of input text

%\documentclass[11pt]{article}
%
%\setlength{\oddsidemargin}{0.0in}
%\setlength{\evensidemargin}{0.0in}
%\setlength{\topmargin}{-0.25in}
%\setlength{\headheight}{0in}
%\setlength{\headsep}{0in}
%\setlength{\textwidth}{6.5in}
%\setlength{\textheight}{9.25in}
%\setlength{\parindent}{0in}
%\setlength{\parskip}{2mm}
%
%\title{The Binomial Distribution}
%\author{Chris Fonnesbeck}
%
%\begin{document}
%
%\maketitle % generates a title based on the information in the preamble
%
%The binomial distribution is a discrete probability distribution used to
%model bounded counts. Specifically, a binomial random variable $x$ represents
%% the $ delimiter marks the start and end of a mathematical expression
%the number of successes in $n$ independent Bernoulli trials, where $p$ is
%the probability of success, and $1-p$ the probability of failure.
%
%% a blank line means a new paragraph
%
%The probability mass function of the binomial distribution is:
%
%\begin{equation}
%    f(x | n,p) = {n \choose x} p^x (1-p)^{n-x} % note the need for braces around the n-x
%\end{equation}
%
%\noindent where $x = 0,1,2,\ldots,n$ and $p \in [0,1]$.
%
%Applications of the binomial distribution include:
%
%\begin{itemize} % this is an unordered list
%    \item Estimation of probabilities of outcomes in any set of success or failure trials
%    \item Estimation of probabilities of outcomes in games of chance
%    \item Sampling of attribute
%\end{itemize}
%
%\end{document}  % required; the document ends here

%Command											Level
%\part{this is a part}							-1
%\chapter{this is a chapter}						0
%\section{this is a section}						1
%\subsection{this is a subsection}				2
%\subsubsection{this is a subsubsection}			3
%\paragraph{this is a paragraph}					4
%\subparagraph{this is a subparagraph}			5

% LaTeX Tutorial I A Primer Indian TeX Users Groups

% Chapter 1: The Basics
%
%\documentclass[14pt]{article}
%\begin{document}
%
%This is my \emph{first} document prepared in \LaTeX.
%
%I typed it on \today.
%
%\begin{center}
%The \TeX nical Institute\\[.75cm]
%Certificate
%\end{center}
%\noindent This is to certify that Mr. N. O. Vice has undergone a course at this institute and is qualified to be a \TeX nician.
%
%\begin{flushright}
%The Director\\
%The \TeX nical Institute.
%\end{flushright}
%
%\end{document}

% Chapter 2: The Document
\documentclass[a4paper]{article}
\begin{document}
\title{My title}
\author{Chen Chen}
\date{\today}
\maketitle

\begin{abstract}
This is my abstracts.
\end{abstract}

\chapter{The Document}

\section{Section}
My first section.
\subsection{subsection}
my subsection

my car is good
\end{document}

